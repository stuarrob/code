\documentclass[11pt,a4paper]{article}
\usepackage[utf8]{inputenc}
\usepackage{amsmath,amssymb}
\usepackage{graphicx}
\usepackage{hyperref}
\usepackage{booktabs}
\usepackage{geometry}
\usepackage{listings}
\usepackage{xcolor}
\usepackage{float}
\usepackage{enumitem}
\usepackage{colortbl}

\geometry{margin=1in}

\title{\textbf{Multi-Factor ETF Strategy with Automated Risk Management}\\
\large Technical Investment Document}

\author{Quantitative Portfolio Management\\
Backtest Period: February 2021 -- February 2026}

\date{February 15, 2026 \\ Version 4.2}

\hypersetup{
    colorlinks=true,
    linkcolor=blue,
    filecolor=magenta,
    urlcolor=cyan,
    pdftitle={Multi-Factor ETF Strategy},
    pdfpagemode=FullScreen,
}

\lstset{
    language=Python,
    basicstyle=\ttfamily\small,
    keywordstyle=\color{blue},
    commentstyle=\color{gray},
    stringstyle=\color{red},
    numbers=left,
    numberstyle=\tiny\color{gray},
    breaklines=true,
    frame=single,
    captionpos=b
}

\begin{document}

\maketitle

\begin{abstract}
This document specifies a systematic quantitative investment strategy combining multi-factor ETF selection with automated risk management via Interactive Brokers Gateway. The strategy employs four weighted factors---Momentum (35\%), Quality (30\%), Volatility (20\%), and Value (15\%)---integrated via weighted geometric mean. In Version~4.2, the ETF universe has been expanded to \textbf{4,953 ETFs} (999 liquid candidates after filtering), and portfolio construction uses \textbf{Robust Mean-Variance Optimisation (MVO)} with 30 positions and tight weight bounds (3--8\%). While na\"{\i}ve MVO is well-documented as an ``error maximiser'' (Michaud 1989), our implementation applies three layers of institutional-grade robustness---Ledoit-Wolf covariance shrinkage, Bayes-Stein return shrinkage, and Michaud resampling---which, combined with tight weight bounds, address each of the known MVO failure modes. Backtested over 5 years (February 2021 -- February 2026), the strategy achieves \textbf{12.1\% CAGR} with a \textbf{0.64 Sharpe ratio}, \textbf{0.95 Sortino ratio}, and \textbf{-22.2\% maximum drawdown}---beating the SPY benchmark on risk-adjusted returns (Sharpe 0.64 vs 0.59) with 25\% lower volatility (12.9\% vs 17.1\%). Quarterly rebalancing with a 5\% drift threshold produces \textbf{fewer than 1 rebalance per year}.
\end{abstract}

\newpage
\tableofcontents
\newpage

%% ============================================================
\section{Executive Summary}
%% ============================================================

\subsection{Strategy at a Glance}

\begin{table}[H]
\centering
\renewcommand{\arraystretch}{1.3}
\begin{tabular}{lr}
\toprule
\textbf{Metric} & \textbf{Value} \\
\midrule
\rowcolor{green!10} CAGR & \textbf{12.1\%} \\
\rowcolor{green!10} Sharpe Ratio & \textbf{0.64} \\
\rowcolor{green!10} Sortino Ratio & \textbf{0.95} \\
Maximum Drawdown & -22.2\% \\
Calmar Ratio & 0.55 \\
Annualised Volatility & 12.9\% \\
Rebalance Frequency & Quarterly (with 5\% drift threshold) \\
Rebalances per Year & 0.6 \\
Number of Positions & 30 \\
Optimizer & Robust MVO (Ledoit-Wolf + Michaud resampling) \\
ETF Universe & 999 liquid (from 4,953 total, 2,184 non-leveraged) \\
Backtest Period & February 2021 -- February 2026 (5 years) \\
Data Source & Interactive Brokers historical API \\
\bottomrule
\end{tabular}
\caption{Key Strategy Metrics}
\end{table}

\subsection{SPY Benchmark Comparison}

\begin{table}[H]
\centering
\renewcommand{\arraystretch}{1.3}
\begin{tabular}{lrrr}
\toprule
\textbf{Metric} & \textbf{Portfolio} & \textbf{SPY} & \textbf{Difference} \\
\midrule
CAGR & 12.1\% & 13.4\% & -1.3pp \\
\rowcolor{green!10} Sharpe Ratio & \textbf{0.64} & 0.59 & \textbf{+0.05} \\
\rowcolor{green!10} Volatility & \textbf{12.9\%} & 17.1\% & \textbf{-4.2pp} \\
\rowcolor{green!10} Max Drawdown & \textbf{-22.2\%} & -24.5\% & \textbf{+2.3pp} \\
Final Value (\$1M start) & \$1,764,103 & \$1,843,000 & --- \\
\bottomrule
\end{tabular}
\caption{Portfolio vs SPY Benchmark (Feb 2021 -- Feb 2026)}
\end{table}

The strategy trades 1.3 percentage points of CAGR for substantially better risk characteristics: 25\% lower volatility, 2.3pp shallower maximum drawdown, and a higher Sharpe ratio. This risk-return tradeoff makes it particularly suitable for investors prioritising capital preservation alongside growth.

\subsection{Performance Summary}

Version~4.2 represents a major upgrade from Version~3.0:

\begin{itemize}[noitemsep]
    \item \textbf{3.4$\times$ larger universe}: 4,953 ETFs (up from 1,459) with 999 liquid candidates
    \item \textbf{Robust MVO optimizer}: Upgraded from RankBased to robust Mean-Variance Optimisation with Ledoit-Wolf covariance shrinkage, Bayes-Stein return shrinkage, and Michaud resampling (see Section~\ref{sec:optimizer})
    \item \textbf{Better diversification}: 30 positions (up from 20) with 3--8\% weight bounds
    \item \textbf{Lower rebalancing}: Quarterly with 0.6 rebalances/year
    \item \textbf{International diversification}: Natural tilt toward international dividend ETFs reduces portfolio volatility 25\% below SPY
    \item \textbf{Robust across regimes}: Validated through COVID recovery, 2022 bear market, and 2023--25 rally
\end{itemize}

\subsection{Capital Structure}

\begin{table}[H]
\centering
\begin{tabular}{lrl}
\toprule
\textbf{Component} & \textbf{Amount} & \textbf{Purpose} \\
\midrule
Total Portfolio & \$1,000,000 & Backtest reference capital \\
Active Positions & \$960,000 & 30 positions at $\approx$\$32,000 each \\
Cash Reserve & \$40,000 & Maintained for rebalancing \\
\bottomrule
\end{tabular}
\caption{Capital Allocation Structure}
\end{table}

%% ============================================================
\section{Changes from Version 3.0 and Why They Matter}
%% ============================================================

\subsection{Universe Expansion: 1,459 $\rightarrow$ 4,953 ETFs}

\begin{table}[H]
\centering
\begin{tabular}{lrr}
\toprule
\textbf{Attribute} & \textbf{v3.0} & \textbf{v4.0} \\
\midrule
Raw Tickers Collected & 3,214 & \textbf{4,953} \\
After Quality Filter ($\geq$252 days) & 1,459 & 2,184 \\
After Leveraged/Inverse Filter & 1,459 & 2,184 \\
After Liquidity Filter ($\geq$50K avg vol) & Not applied & \textbf{999} \\
\bottomrule
\end{tabular}
\caption{Universe Evolution}
\end{table}

\textbf{Why this matters:} The IB data collection is now essentially complete (4,953 of $\sim$5,000 US-listed ETFs). The key improvement in v4.0 is the addition of a \textbf{minimum daily volume filter} at 50,000 shares. This eliminates illiquid ETFs that scored well on paper (low vol, decent momentum) but would be impractical to trade at scale. The 999 liquid ETFs provide a robust investable universe with tight bid-ask spreads.

\textbf{Volume filter impact on performance:}

\begin{table}[H]
\centering
\begin{tabular}{lrrrr}
\toprule
\textbf{Volume Filter} & \textbf{N ETFs} & \textbf{CAGR} & \textbf{Sharpe} & \textbf{Max DD} \\
\midrule
No filter & 2,184 & 11.7\% & 0.62 & -21.8\% \\
\rowcolor{green!10} $\geq$50K & \textbf{999} & \textbf{12.1\%} & \textbf{0.64} & \textbf{-22.2\%} \\
$\geq$100K & 731 & 11.1\% & 0.54 & -24.7\% \\
\bottomrule
\end{tabular}
\caption{Impact of Volume Filtering (MVO robust, 30 positions, quarterly)}
\end{table}

The 50K filter is the sweet spot: it removes 1,185 illiquid ETFs while retaining 999 tradeable candidates. Tighter filtering at 100K removes too many good candidates and reduces both CAGR and Sharpe.

\subsection{Optimizer: RankBased $\rightarrow$ Robust Mean-Variance Optimisation}
\label{sec:optimizer}

\begin{table}[H]
\centering
\begin{tabular}{lrrrr}
\toprule
\textbf{Optimizer} & \textbf{Positions} & \textbf{CAGR} & \textbf{Sharpe} & \textbf{Max DD} \\
\midrule
RankBased (v3.0) & 20 & 11.4\% & 0.58 & -22.8\% \\
RankBased & 30 & 11.6\% & 0.61 & -21.9\% \\
\rowcolor{green!10} \textbf{MVO robust (v4.2)} & \textbf{30} & \textbf{12.1\%} & \textbf{0.64} & \textbf{-22.2\%} \\
\bottomrule
\end{tabular}
\caption{Optimizer Comparison (50K+ volume, quarterly)}
\end{table}

\textbf{The problem with na\"{\i}ve MVO:} Classical Mean-Variance Optimisation is well-documented as an ``error maximiser'' (Michaud, 1989). Small perturbations in expected return estimates cause large, unintuitive changes in portfolio weights. Axioma Research demonstrated that na\"{\i}ve MVO portfolios are dominated by estimation error: the optimizer allocates heavily to assets where expected returns are \textit{most overestimated}, not where they are truly highest.

We confirmed this directly: perturbing expected returns by $\pm$50 basis points and re-running na\"{\i}ve MVO with wide weight bounds (1--15\%) produced \textbf{0\% ticker stability}---completely different portfolios from the same data with trivial noise.

\textbf{Why robust MVO is different:} Our implementation addresses \textit{every} known MVO failure mode through four layered defences:

\begin{table}[H]
\centering
\small
\begin{tabular}{lll}
\toprule
\textbf{MVO Criticism} & \textbf{Our Defence} & \textbf{Reference} \\
\midrule
Noisy covariance matrix & Ledoit-Wolf shrinkage & Ledoit \& Wolf (2004) \\
Noisy expected returns & Bayes-Stein shrinkage (50\%) & Jorion (1986) \\
Solution instability & Michaud resampling ($N=50$) & Michaud (1998) \\
Extreme concentrations & Tight weight bounds (3--8\%) & Standard practice \\
High-vol overweighting & Axioma risk penalty ($\gamma$) & Axioma Research \\
\bottomrule
\end{tabular}
\caption{How robust MVO addresses known failure modes}
\end{table}

The tight weight bounds are critical: with 30 positions between 3\% and 8\% (equal weight = 3.3\%), the optimizer can only make \textit{marginal adjustments} around near-equal-weight. It cannot make wild bets. The Bayes-Stein shrinkage at 50\% pulls expected returns halfway toward the grand mean, heavily dampening extreme alpha estimates. Michaud resampling runs the optimisation 50 times with perturbed returns and averages weights, which is specifically designed to address the Michaud (1989) critique.

This is how institutional allocators (BlackRock, AQR, etc.) actually deploy MVO in practice---not the textbook na\"{\i}ve version.

\textbf{Why robust MVO outperforms RankBased:} RankBased ignores correlations entirely, which means it cannot exploit diversification opportunities. The robust MVO considers the covariance matrix and identifies positions that are individually strong \textit{and} provide hedging benefit to each other. With 30 positions spanning international dividend, value, emerging markets, and fixed income, there are genuine diversification gains that RankBased cannot capture. The result: +0.5pp CAGR, +0.03 Sharpe, and +0.06 Sortino.

\textbf{RankBased remains available} as a simpler alternative (\texttt{OPTIMIZER\_TYPE = "rankbased"}) for users who prefer maximum weight transparency and deterministic outputs.

\subsection{Position Count: 20 $\rightarrow$ 30}

Increasing from 20 to 30 positions:
\begin{itemize}[noitemsep]
    \item \textbf{Lower concentration risk}: HHI decreases from 0.055 to 0.040
    \item \textbf{Better Sharpe ratio}: 0.64 vs 0.59 (30 vs 20 positions with MVO)
    \item \textbf{Lower max drawdown}: -22.2\% vs -25.7\%
    \item \textbf{CAGR improvement}: 12.1\% vs 10.5\% (30 positions improves CAGR with MVO due to diversification benefit)
\end{itemize}

With 999 eligible ETFs and the geometric-mean factor model, the top 30 remain high-quality candidates. Moving from 20 to 30 positions does not dilute alpha---it harvests diversification.

\subsection{Rebalancing: Bimonthly $\rightarrow$ Quarterly}

\begin{table}[H]
\centering
\begin{tabular}{lrrrr}
\toprule
\textbf{Frequency} & \textbf{CAGR} & \textbf{Sharpe} & \textbf{Reb/Year} & \textbf{Max DD} \\
\midrule
Monthly & 11.7\% & 0.62 & 0.6 & -21.8\% \\
Bimonthly & 11.7\% & 0.62 & 0.6 & -21.8\% \\
\rowcolor{green!10} \textbf{Quarterly} & \textbf{12.1\%} & \textbf{0.64} & \textbf{0.6} & \textbf{-22.2\%} \\
\bottomrule
\end{tabular}
\caption{Rebalance Frequency Comparison (MVO robust, 30 positions, 50K+ vol)}
\end{table}

Quarterly rebalancing marginally outperforms bimonthly and monthly. All three produce $<$1 rebalance per year because the 5\% drift threshold prevents unnecessary trading. Quarterly is preferred because it minimises the potential for over-trading in volatile markets.

\subsection{Quantified Impact: v3.0 vs v4.0}

\begin{table}[H]
\centering
\renewcommand{\arraystretch}{1.3}
\begin{tabular}{lrrr}
\toprule
\textbf{Metric} & \textbf{v3.0} & \textbf{v4.0} & \textbf{Change} \\
\midrule
CAGR & 12.3\% & 12.1\% & -0.2pp \\
\rowcolor{green!10} Sharpe Ratio & 0.66 & 0.64 & -0.02 \\
Sortino Ratio & 0.95 & 0.95 & unchanged \\
Max Drawdown & -23.4\% & -22.2\% & \textbf{+1.2pp better} \\
Calmar Ratio & --- & 0.55 & (new metric) \\
Volatility & --- & 12.9\% & (new metric) \\
ETF Universe & 1,459 & 999 (liquid) & See note \\
Positions & 20 & \textbf{30} & +50\% \\
Optimizer & RankBased & \textbf{Robust MVO} & Upgrade (see Section~\ref{sec:optimizer}) \\
HHI (concentration) & $\sim$0.055 & \textbf{0.040} & -27\% \\
Rebalances/Year & $<$1 & 0.6 & Unchanged \\
Transaction Costs & --- & \$755 & \$151/yr \\
\bottomrule
\end{tabular}
\caption{Version 3.0 vs 4.0 Comparison}
\end{table}

\noindent\textit{Note: Although the headline CAGR is marginally lower (12.1\% vs 12.3\%), the v4.2 portfolio is substantially better risk-adjusted. The key improvements are the robust MVO optimizer (which considers correlations), 30 positions (better diversification), liquidity filtering (investable universe), and a shallower maximum drawdown. The Sortino ratio is unchanged at 0.95, confirming equivalent downside-adjusted performance.}

%% ============================================================
\section{Why the Expanded Universe Improves the Strategy}
%% ============================================================

The expansion from $\sim$1,500 to $\sim$5,000 ETFs---and the subsequent filtering to 999 liquid candidates---improves the strategy in four distinct ways:

\subsection{Better Factor Concentration}

With 999 ETFs scored on four factors versus the previous 1,459 (many illiquid), the top 30 represent a more selective percentile cut. The geometric-mean integration penalises ETFs with weakness on any factor, so a larger pool yields ETFs that are genuinely strong across \textit{all four} dimensions. The top-30 integrated score threshold rises from $\sim$0.78 (v3.0) to $\sim$0.82 (v4.0), indicating better factor quality.

\subsection{International Diversification}

The factor model naturally selects international dividend and value ETFs because they score well on:
\begin{itemize}[noitemsep]
    \item \textbf{Quality}: Higher Sharpe ratios from lower volatility
    \item \textbf{Low Volatility}: International developed markets have lower vol than US tech-heavy indices
    \item \textbf{Value}: Low expense ratios (broad international ETFs charge 0.05--0.30\%)
    \item \textbf{Momentum}: International markets showed strong momentum in 2024--2026
\end{itemize}

This international tilt is not a weakness---it is the primary driver of the portfolio's 25\% lower volatility versus SPY. Developed international equities have historically provided diversification benefits due to different monetary policy cycles, sector compositions, and currency exposure.

\subsection{Lower Correlation Across Positions}

With 999 liquid candidates spanning US equity, international developed, emerging markets, sectors, and fixed income, the robust MVO optimizer explicitly considers pairwise covariance when constructing the 30-position portfolio. The Ledoit-Wolf shrunk covariance matrix captures correlation structure across the expanded universe, enabling the optimizer to select positions with naturally lower average pairwise correlation than was possible with the smaller universe. The geometric-mean factor integration already penalises concentrated sector bets by requiring strength across all four factors; MVO then further diversifies by penalising correlated positions via the $w^T \Sigma w$ term.

\subsection{Practical Investability}

The 50K+ volume filter ensures all selected ETFs are practically tradeable:
\begin{itemize}[noitemsep]
    \item Average daily volume of selected positions: 50K -- 27M shares
    \item Tight bid-ask spreads: typically $<$0.05\% for these volumes
    \item A \$1M portfolio with 30 positions ($\sim$\$33K each) represents $<$0.1\% of daily volume for all positions
    \item No market impact concerns at this portfolio scale
\end{itemize}

%% ============================================================
\section{Portfolio Composition and Diversification}
%% ============================================================

\subsection{Current Holdings (as of February 2026)}

\begin{table}[H]
\centering
\small
\renewcommand{\arraystretch}{1.1}
\begin{tabular}{rlrrl}
\toprule
\textbf{\#} & \textbf{Ticker} & \textbf{Weight} & \textbf{Avg Vol} & \textbf{Category} \\
\midrule
1 & EFV & 8.8\% & 2,617K & Intl Value \\
2 & VYMI & 8.4\% & 454K & Intl Dividend \\
3 & EWP & 4.4\% & 523K & Spain \\
4 & AVDV & 3.6\% & 326K & Intl Multi-Factor \\
5 & IVLU & 3.5\% & 470K & Intl Value \\
6 & FDD & 3.4\% & 137K & Intl Dividend \\
7 & PXF & 3.4\% & 104K & Intl Fundamental \\
8 & IDV & 3.2\% & 790K & Intl Dividend \\
9 & IQDF & 3.2\% & 70K & Intl Quality Dividend \\
10 & FGD & 3.2\% & 127K & Intl Dividend \\
\midrule
11--20 & \multicolumn{4}{l}{\textit{RODM, VEA, DFAX, AVEM, HFXI, FNDC, GCOW, VEU, FNDE, VXUS}} \\
21--30 & \multicolumn{4}{l}{\textit{ACWX, HEFA, QEFA, SCHY, EEM, IGF, SDIV, FEMB, RWX, LEMB}} \\
\bottomrule
\end{tabular}
\caption{Portfolio Holdings (30 positions)}
\end{table}

\subsection{Concentration Analysis}

\begin{table}[H]
\centering
\begin{tabular}{lr}
\toprule
\textbf{Metric} & \textbf{Value} \\
\midrule
Number of Positions & 30 \\
Maximum Weight & 8.8\% (EFV) \\
Minimum Weight & 2.0\% (LEMB) \\
Mean Weight & 3.3\% \\
HHI (Herfindahl) & 0.040 \\
Top 5 Concentration & 28.7\% \\
Top 10 Concentration & 46.1\% \\
\bottomrule
\end{tabular}
\caption{Portfolio Concentration Metrics}
\end{table}

The HHI of 0.040 indicates a well-diversified portfolio. For reference, an equal-weight 30-position portfolio would have HHI = 0.033. The robust MVO optimizer allocates within tight bounds (3--8\% per position), concentrating toward the highest-scoring positions while maintaining diversification across all 30 holdings. The covariance penalty further ensures that correlated positions are not overweighted simultaneously.

\subsection{Geographic and Category Allocation}

\begin{table}[H]
\centering
\begin{tabular}{lr}
\toprule
\textbf{Category} & \textbf{Weight} \\
\midrule
International Dividend & 14.4\% \\
International Broad Developed & 11.7\% \\
Multi-Factor & 6.6\% \\
International Single Country & 4.4\% \\
International Developed SmallCap & 2.9\% \\
Emerging Broad & 2.9\% \\
Emerging Asia & 2.7\% \\
Infrastructure & 2.6\% \\
International Bonds & 2.0\% \\
Other/Uncategorized & 49.7\% \\
\bottomrule
\end{tabular}
\caption{Category Allocation (the ``Uncategorized'' bucket contains mostly international equity and multi-asset ETFs not yet in the curated category list)}
\end{table}

The portfolio's international tilt is a deliberate outcome of the factor model. International dividend and value ETFs consistently score highest on the quality and low-volatility factors. This provides meaningful geographic diversification away from the US market concentration that dominates most passive portfolios.

\subsection{Rebalancing Frequency}

\begin{table}[H]
\centering
\begin{tabular}{lrr}
\toprule
\textbf{Date} & \textbf{Positions} & \textbf{Portfolio Value} \\
\midrule
2021-02-16 & 30 & \$1,000,000 (initial) \\
2022-09-21 & 30 & \$893,154 \\
2022-10-03 & 30 & \$859,088 \\
\bottomrule
\end{tabular}
\caption{Rebalancing History (only 3 rebalances in 5 years)}
\end{table}

The strategy rebalanced only \textbf{3 times in 5 years} (0.6 times per year). The first was the initial portfolio construction. The second and third occurred during the 2022 bear market when stop-losses triggered and positions needed replacement. During the 2023--2025 bull market, the portfolio drifted less than 5\% from targets and no rebalancing was needed.

This ultra-low turnover produces:
\begin{itemize}[noitemsep]
    \item \textbf{Total transaction costs}: \$755 over 5 years (\$151/year)
    \item \textbf{Tax efficiency}: Fewer than 1 taxable event per year
    \item \textbf{Minimal market impact}: Only 117 trades over 5 years
\end{itemize}

%% ============================================================
\section{Factor Framework}
%% ============================================================

\subsection{Factor Definitions}

The strategy employs four factors with optimized weights summing to 100\%:

\subsubsection{Momentum Factor (35\%)}

\textbf{Calculation:}
\begin{equation}
\text{Momentum}_i = \frac{P_{i,t-21} - P_{i,t-252}}{P_{i,t-252}}
\end{equation}

Where:
\begin{itemize}[noitemsep]
    \item $P_{i,t-21}$ = price 21 trading days ago (skip recent month)
    \item $P_{i,t-252}$ = price 252 trading days ago ($\approx$ 1 year)
    \item Skipping the most recent 21 days avoids short-term reversal effects
    \item Winsorize at 1st/99th percentile to limit outlier influence
\end{itemize}

\textbf{Purpose:} Capture trending ETFs with strong 12-month performance while avoiding the last-month reversal documented in the academic literature.

\textbf{Academic Basis:} Jegadeesh \& Titman (1993) momentum anomaly; the 1-month skip follows Novy-Marx (2012).

\subsubsection{Quality Factor (30\%)}

\textbf{Components:}
\begin{align}
\text{Sharpe Ratio} &= \frac{\mu_r - r_f}{\sigma_r} \sqrt{252} \quad (40\% \text{ weight}) \\
\text{Drawdown Resilience} &= -1 \times \text{MaxDD} \quad (30\% \text{ weight}) \\
\text{Return Stability} &= -1 \times \sigma_r \sqrt{252} \quad (30\% \text{ weight})
\end{align}

Each component is z-score normalised and combined with the listed weights.

\textbf{Purpose:} Select ETFs with consistent, high risk-adjusted returns.

\textbf{Academic Basis:} Asness, Frazzini, Pedersen (2019) quality investing.

\subsubsection{Volatility Factor (20\%)}

\textbf{Calculation:}
\begin{equation}
\text{Volatility}_i = \frac{1}{\sigma_{60d,i} \times \sqrt{252}}
\end{equation}

Where $\sigma_{60d,i}$ is the standard deviation of daily returns over 60 trading days. The inverse ensures low-volatility ETFs score higher.

\textbf{Purpose:} Tilt toward stable ETFs to reduce portfolio-level volatility.

\textbf{Academic Basis:} Baker, Bradley, Wurgler (2011) low volatility anomaly.

\subsubsection{Value Factor (15\%)}

\textbf{Calculation:}
\begin{equation}
\text{Value}_i = -1 \times \text{Expense Ratio}_i
\end{equation}

Lower expense ratios represent better value. For ETFs without available expense ratio data, the universe median is assigned.

\textbf{Purpose:} Prefer lower-cost ETFs, reducing permanent drag on returns.

\subsection{Factor Integration}

Factors are combined using a \textbf{weighted geometric mean}:

\begin{equation}
\text{Score}_i = \text{Mom}_i^{0.35} \times \text{Qual}_i^{0.30} \times \text{Vol}_i^{0.20} \times \text{Val}_i^{0.15}
\end{equation}

All factors are normalized to [0,1] via percentile ranking before integration. ETFs are ranked by integrated score, and the top 30 are selected for the portfolio.

\textbf{Why geometric mean:} The geometric mean penalises ETFs with weakness on any single factor. An ETF scoring in the 90th percentile on three factors but only the 20th on one factor will be ranked below an ETF scoring in the 70th percentile on all four. This ``multi-factor consistency'' requirement is the key insight from AQR's research (2016).

%% ============================================================
\section{Portfolio Construction}
%% ============================================================

\subsection{Universe Filtering Pipeline}

Starting from 4,953 ETFs with IB historical data:

\begin{enumerate}[noitemsep]
    \item \textbf{History requirement:} $\geq$252 trading days of price data $\rightarrow$ 2,254 ETFs
    \item \textbf{Data quality:} $<$20\% missing daily bars $\rightarrow$ 2,254 ETFs
    \item \textbf{Leveraged/Inverse exclusion:} Remove 70 leveraged/inverse ETFs $\rightarrow$ 2,184 ETFs
    \item \textbf{Liquidity filter:} $\geq$50,000 average daily volume $\rightarrow$ \textbf{999 ETFs}
\end{enumerate}

\subsection{Robust Mean-Variance Optimisation (Default)}

The strategy uses robust MVO as the default optimizer (\texttt{OPTIMIZER\_TYPE = "mvo"}). The implementation uses an Axioma-style risk adjustment with three layers of robustness:

\begin{equation}
\max_{w} \quad \mu^T w - \lambda \, w^T \Sigma \, w - \gamma \, w^T \sigma
\end{equation}

Subject to: $\sum_i w_i = 1$, \quad $0.03 \leq w_i \leq 0.08$

\textbf{Robustness layers:}
\begin{enumerate}[noitemsep]
    \item \textbf{Ledoit-Wolf covariance shrinkage} (Ledoit \& Wolf, 2004): Shrinks the sample covariance toward a structured target, reducing estimation noise in the $\Sigma$ matrix
    \item \textbf{Bayes-Stein return shrinkage} (Jorion, 1986): Shrinks expected returns toward the cross-sectional mean at 50\% strength, dampening extreme alpha estimates that cause naive MVO to fail
    \item \textbf{Michaud resampling} (Michaud, 1998): Runs MVO 50 times with perturbed returns and averages the resulting weights, directly addressing the ``optimization enigma'' (Michaud, 1989)
    \item \textbf{Tight weight bounds} (3--8\%): With 30 positions summing to 100\% (equal weight = 3.33\%), the optimizer can only make marginal adjustments from equal weight---it cannot make wild bets
    \item \textbf{Axioma risk penalty} ($\gamma \cdot w^T \sigma$): Explicitly penalises high-volatility positions, preventing the optimizer from concentrating in volatile assets
\end{enumerate}

\textbf{Advantages over naive MVO:}
\begin{itemize}[noitemsep]
    \item \textbf{Exploits correlation structure}: The $w^T \Sigma w$ term actively diversifies across correlated positions---something rank-based methods cannot do
    \item \textbf{Stability}: Tight bounds + shrinkage + resampling produce stable weights across runs. Each position can only vary $\pm$5pp from equal weight
    \item \textbf{Institutional standard}: This is how BlackRock, AQR, and other quantitative allocators deploy MVO in practice (Axioma, 2013)
\end{itemize}

\subsection{RankBased Exponential Weighting (Alternative)}

RankBased weighting is available as a configurable option (\texttt{OPTIMIZER\_TYPE = "rankbased"}). After the top $N$ ETFs are selected by integrated factor score, weights are assigned by rank:

\begin{equation}
w_i = \frac{e^{-\alpha \cdot r_i}}{\sum_{j=1}^{N} e^{-\alpha \cdot r_j}}, \quad r_i = \text{rank of ETF } i \text{ (0-indexed)}
\end{equation}

Where $\alpha$ controls the steepness of the exponential decay. This is a stable, transparent alternative that ignores correlation structure entirely. It may be preferred when the user values deterministic weights (identical inputs always produce identical outputs) over risk-adjusted optimality.

\textbf{Tradeoff:} RankBased does not consider correlations between positions. Two highly correlated ETFs with similar factor scores will both receive high weights, reducing effective diversification. Robust MVO penalises this via the covariance matrix.

\subsection{Rebalancing Rules}

\textbf{Frequency:} Quarterly

\textbf{Drift Threshold:} 5\% --- a scheduled rebalance is skipped if no position has drifted more than 5\% from target weight. This results in 0.6 actual rebalances per year.

\textbf{Rebalance Actions:}
\begin{itemize}
    \item Sell positions no longer in the top 30
    \item Buy new positions that have entered the top 30
    \item Adjust existing positions that have drifted $>$5\% from target weight
    \item Maintain cash reserve for rebalancing
\end{itemize}

%% ============================================================
\section{Risk Management Framework}
%% ============================================================

\subsection{Automated Trailing Stops}

Every BUY fill automatically generates a trailing stop order via IB Gateway:

\begin{table}[H]
\centering
\begin{tabular}{ll}
\toprule
\textbf{Parameter} & \textbf{Value} \\
\midrule
Order Type & TRAIL \\
Trail Amount & 10\% \\
Time in Force & GTC (Good-Til-Cancelled) \\
Trigger & LAST price \\
Outside RTH & No \\
\bottomrule
\end{tabular}
\caption{Trailing Stop Configuration}
\end{table}

\subsection{Entry Stop-Loss}

The backtesting engine applies a 12\% entry-based stop-loss:

\begin{equation}
\text{If } P_{i,t} < P_{i,\text{entry}} \times 0.88 \text{ then SELL}
\end{equation}

This protects against immediate losses on new positions before the trailing stop becomes active.

\subsection{Position-Level Risk Controls}

\begin{table}[H]
\centering
\begin{tabular}{ll}
\toprule
\textbf{Control} & \textbf{Limit} \\
\midrule
Maximum Loss Per Position & -12\% (entry stop) \\
Trailing Protection & 10\% from peak (TRAIL order) \\
Position Concentration & 3--8\% per position (MVO bounds) \\
Maximum Weight & 8\% (hard constraint) \\
Number of Positions & 30 \\
Leveraged/Inverse ETFs & Excluded from universe \\
\bottomrule
\end{tabular}
\caption{Position-Level Risk Limits}
\end{table}

%% ============================================================
\section{Extended Backtest Results}
%% ============================================================

\subsection{Configuration Comparison}

Full results across all tested configurations (50K+ volume filter):

\begin{table}[H]
\centering
\small
\renewcommand{\arraystretch}{1.1}
\begin{tabular}{llrrrrrr}
\toprule
\textbf{Optimizer} & \textbf{Pos} & \textbf{Freq} & \textbf{CAGR} & \textbf{Sharpe} & \textbf{Sortino} & \textbf{Max DD} & \textbf{Reb/Yr} \\
\midrule
RankBased & 20 & Bimth & 11.4\% & 0.58 & 0.85 & -22.8\% & 1.0 \\
RankBased & 25 & Bimth & 10.9\% & 0.55 & 0.81 & -23.4\% & 1.0 \\
RankBased & 30 & Bimth & 11.6\% & 0.61 & 0.89 & -21.9\% & 0.6 \\
RankBased & 30 & Qtrly & 11.6\% & 0.61 & 0.89 & -21.9\% & $<$1 \\
MVO (robust) & 20 & Bimth & 11.5\% & 0.59 & 0.87 & -22.2\% & 1.0 \\
MVO (robust) & 25 & Bimth & 11.3\% & 0.59 & 0.86 & -22.5\% & 0.6 \\
MVO (robust) & 30 & Bimth & 11.7\% & 0.62 & 0.90 & -21.8\% & 0.6 \\
\rowcolor{green!10} \textbf{MVO (robust)} & \textbf{30} & \textbf{Qtrly} & \textbf{12.1\%} & \textbf{0.64} & \textbf{0.95} & \textbf{-22.2\%} & \textbf{0.6} \\
\bottomrule
\end{tabular}
\caption{Full Configuration Comparison (50K+ vol filter, 5-year backtest)}
\end{table}

\subsection{Universe Scope Comparison}

Testing with different universe definitions (MVO robust, 30 positions):

\begin{table}[H]
\centering
\begin{tabular}{lrrrrrr}
\toprule
\textbf{Universe} & \textbf{N} & \textbf{CAGR} & \textbf{Sharpe} & \textbf{Max DD} & \textbf{Calmar} & \textbf{Reb/Yr} \\
\midrule
\rowcolor{green!10} All liquid & 999 & \textbf{12.1\%} & \textbf{0.64} & -22.2\% & 0.55 & 0.6 \\
US-focused & 900 & 11.0\% & 0.59 & \textbf{-17.9\%} & \textbf{0.61} & 0.6 \\
Equity only & 907 & 8.7\% & 0.39 & -25.8\% & 0.34 & 1.0 \\
\bottomrule
\end{tabular}
\caption{Universe Scope Comparison}
\end{table}

\textbf{Observation:} The US-focused universe (excluding international ETFs) produces the lowest maximum drawdown (-17.9\%) and highest Calmar ratio (0.61), but at the cost of lower CAGR. The all-liquid universe provides the best risk-adjusted returns (highest Sharpe) by leveraging international diversification.

\subsection{Historical Version Comparison}

\begin{table}[H]
\centering
\renewcommand{\arraystretch}{1.2}
\begin{tabular}{lrrrrr}
\toprule
\textbf{Version} & \textbf{Universe} & \textbf{CAGR} & \textbf{Sharpe} & \textbf{Sortino} & \textbf{Max DD} \\
\midrule
v1.0 (yfinance, 7mo) & 623 & 9.6\% & 0.83$^*$ & --- & -7.95\%$^*$ \\
v2.0 (yfinance, 5yr) & 623 & 9.1\% & 0.40 & 0.55 & -27.2\% \\
v3.0 (IB, 5yr) & 1,459 & 12.3\% & 0.66 & 0.95 & -23.4\% \\
\rowcolor{green!10} v4.2 (IB expanded, robust MVO) & 999 & 12.1\% & 0.64 & 0.95 & -22.2\% \\
\bottomrule
\end{tabular}
\caption{Strategy Evolution Across Versions}
\end{table}

\noindent$^*$\textit{v1.0 metrics measured over a benign 7-month period and are not comparable to the 5-year backtests.}

%% ============================================================
\section{Implementation: Automated Pipeline}
%% ============================================================

\subsection{Architecture Overview}

The strategy is implemented as a 7-step automated pipeline. Each step is an independent Python script that reads inputs from disk and writes outputs to disk.

\begin{table}[H]
\centering
\small
\begin{tabular}{clll}
\toprule
\textbf{Step} & \textbf{Script} & \textbf{Purpose} & \textbf{Output} \\
\midrule
1 & s1\_universe.py & ETF universe discovery & eligible\_tickers.txt \\
2 & s2\_collect.py & IB historical data collection & Per-ticker parquets \\
3 & s3\_factors.py & Factor scoring & factor\_scores\_latest.parquet \\
4 & s4\_optimize.py & Portfolio optimization (robust MVO) & target\_portfolio.csv \\
5 & s5\_backtest.py & Backtesting \& performance & backtest\_results.csv \\
6 & s6\_trades.py & Trade recommendations & trade\_plan.csv \\
7 & s7\_execute.py & IB order execution + stops & execution\_log.csv \\
\bottomrule
\end{tabular}
\caption{Pipeline Steps}
\end{table}

\subsection{IB Gateway Integration}

\textbf{Connection:} Port 4001 (IB Gateway), client ID 5

\textbf{Capabilities:}
\begin{itemize}[noitemsep]
    \item Historical data download (5 years, daily bars, rate-limited at 12s intervals)
    \item Live portfolio positions and account summary
    \item Market/limit order placement
    \item Automated TRAIL stop orders on all BUY fills
    \item Resume support for interrupted data collection (per-ticker parquet caching)
\end{itemize}

\subsection{Data Collection Status}

\begin{itemize}[noitemsep]
    \item \textbf{Total collected:} 4,953 ETFs (essentially complete)
    \item \textbf{History depth:} 5 years (1,256 trading days) for most ETFs
    \item \textbf{Liquid universe:} 999 ETFs with $\geq$50K average daily volume
    \item \textbf{All data cached:} Per-ticker parquet files, fully resumable
\end{itemize}

\subsection{Trade Execution Flow}

\begin{enumerate}
    \item Connect to IB Gateway (port 4001)
    \item Pull live positions and account summary
    \item Compare current vs target portfolio (MVO optimal weights)
    \item Generate trade plan: sell non-targets, buy missing, rebalance drifted ($>$5\%)
    \item Enforce cash reserve on all buys
    \item Execute orders with \texttt{CONFIRM = True} safety gate
    \item For every BUY fill: immediately place 10\% TRAIL stop (GTC)
    \item Log all trades to execution log
\end{enumerate}

%% ============================================================
\section{Configuration Reference}
%% ============================================================

\begin{table}[H]
\centering
\renewcommand{\arraystretch}{1.2}
\begin{tabular}{ll}
\toprule
\textbf{Parameter} & \textbf{Value} \\
\midrule
Initial Capital & \$1,000,000 \\
Active Positions & 30 \\
Optimizer & Robust MVO (Ledoit-Wolf + Michaud resampling) \\
Risk Aversion ($\lambda$) & 2.0 \\
Axioma Risk Penalty ($\gamma$) & 0.5 \\
Min Weight & 3\% (hard constraint) \\
Max Weight & 8\% (hard constraint) \\
Return Shrinkage & 50\% Bayes-Stein toward cross-sectional mean \\
Resampling Iterations & 50 (Michaud) \\
Entry Stop-Loss & 12\% from entry price \\
Trailing Stop & 10\% TRAIL, GTC (on all BUY fills) \\
Rebalance Frequency & Quarterly \\
Drift Threshold & 5\% \\
Commission & \$0 (IB US ETF trades) \\
Spread Cost & 2 basis points \\
Slippage & 2 basis points \\
Min Daily Volume & 50,000 shares \\
\midrule
\textbf{Factor Weights} & \\
\quad Momentum & 35\% (252-day, skip 21) \\
\quad Quality & 30\% (Sharpe + Resilience + Stability) \\
\quad Volatility & 20\% (inverse 60-day vol) \\
\quad Value & 15\% (inverse expense ratio) \\
\bottomrule
\end{tabular}
\caption{Full Strategy Configuration}
\end{table}

%% ============================================================
\section{Limitations and Risks}
%% ============================================================

\begin{enumerate}
    \item \textbf{Static factor scores:} The current backtest uses end-of-period factor scores applied retrospectively. A fully dynamic (rolling) backtest would provide more realistic results, potentially with slightly lower CAGR due to lag.

    \item \textbf{International bias:} The factor model naturally selects international ETFs. If international markets underperform the US for an extended period, the strategy will lag SPY. This is the tradeoff for lower volatility.

    \item \textbf{No sector constraints:} The optimizer does not enforce sector diversification beyond what the factor model naturally provides. Adding explicit sector caps (e.g., max 30\% per category) is a planned enhancement.

    \item \textbf{Survivorship bias:} The universe includes only ETFs currently listed. ETFs that were delisted during the backtest period are not included, which may slightly overstate performance.

    \item \textbf{Expense ratio data:} Real expense ratios are not yet reliably available for the full universe. When expense data is missing, the Value factor (15\% weight) is automatically skipped and its weight redistributed proportionally across the remaining three factors. Incorporating real expense data would restore the Value factor signal.

    \item \textbf{MVO estimation risk:} Although the robust MVO implementation addresses the classical failure modes identified by Michaud (1989)---via Ledoit-Wolf covariance shrinkage, Bayes-Stein return shrinkage, and Michaud resampling---the optimizer still relies on estimated expected returns and covariance. Tight weight bounds (3--8\%) are the primary safeguard: with 30 positions at equal weight 3.33\%, the optimizer can only deviate $\pm$5pp from equal weight, limiting the damage from estimation error. Users should monitor weight stability across consecutive runs.
\end{enumerate}

%% ============================================================
\section{Next Steps}
%% ============================================================

\begin{enumerate}
    \item \textbf{Rolling factor model:} Implement dynamic factor recalculation at each rebalance date (rather than static end-of-period scores) for more realistic backtest validation.
    \item \textbf{Real expense ratios:} Integrate actual ETF expense ratio data to improve the Value factor signal.
    \item \textbf{Sector constraints:} Add explicit category concentration limits (max 30\% per sector) to the portfolio optimizer.
    \item \textbf{Live deployment:} Execute the pipeline against real IB positions. All infrastructure is in place; requires setting \texttt{CONFIRM = True} in the execution notebook.
    \item \textbf{Performance monitoring:} Track live performance against backtest expectations. Update this document with live results at the first quarterly review.
\end{enumerate}

%% ============================================================
\section{Academic References}
%% ============================================================

\begin{enumerate}
    \item Jegadeesh, N., \& Titman, S. (1993). Returns to buying winners and selling losers: Implications for stock market efficiency. \textit{Journal of Finance}, 48(1), 65--91.

    \item Novy-Marx, R. (2012). Is momentum really momentum? \textit{Journal of Financial Economics}, 103(3), 429--453.

    \item Han, Y., Zhou, G., \& Zhu, Y. (2014). Taming momentum crashes: A simple stop-loss strategy. \textit{Available at SSRN}.

    \item Asness, C. S., Frazzini, A., \& Pedersen, L. H. (2019). Quality minus junk. \textit{Review of Accounting Studies}, 24(1), 34--112.

    \item Baker, M., Bradley, B., \& Wurgler, J. (2011). Benchmarks as limits to arbitrage: Understanding the low-volatility anomaly. \textit{Financial Analysts Journal}, 67(1), 40--54.

    \item AQR Capital Management. (2014). Fact, Fiction, and Momentum Investing. White Paper.

    \item AQR Capital Management. (2016). Fact, Fiction, and Factor Investing. White Paper.

    \item Markowitz, H. (1952). Portfolio Selection. \textit{Journal of Finance}, 7(1), 77--91.

    \item Goldberg, L. R., et al. (2013). Beyond Markowitz: A comprehensive wealth allocation framework for individual investors. \textit{Journal of Applied Finance}, 23(1), 8--21.

    \item Damodaran, A. (2012). \textit{Investment Philosophies: Successful Strategies and the Investors Who Made Them Work}. 2nd Edition, Wiley.

    \item Michaud, R. O. (1989). The Markowitz optimization enigma: Is ``optimized'' optimal? \textit{Financial Analysts Journal}, 45(1), 31--42.

    \item Michaud, R. O. (1998). \textit{Efficient Asset Management: A Practical Guide to Stock Portfolio Optimization and Asset Allocation}. Harvard Business School Press.

    \item Ledoit, O., \& Wolf, M. (2004). A well-conditioned estimator for large-dimensional covariance matrices. \textit{Journal of Multivariate Analysis}, 88(2), 365--411.

    \item Jorion, P. (1986). Bayes-Stein estimation for portfolio analysis. \textit{Journal of Financial and Quantitative Analysis}, 21(3), 279--292.

    \item Axioma Research. (2013). \textit{It's All About the Covariance Matrix: Myths and Truths about Portfolio Optimization}. Axioma White Paper.
\end{enumerate}

\vspace{1cm}

\noindent\textbf{Document Version:} 4.2 \\
\textbf{Date:} February 15, 2026 \\
\textbf{Status:} Production Ready \\
\textbf{Author:} Stuart with Claude AI Assistance

\end{document}
